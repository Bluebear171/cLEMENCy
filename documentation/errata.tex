\documentclass[oneside]{book}
\usepackage{tabu}
\usepackage{float}
\usepackage{xcolor}
\usepackage{colortbl}
\usepackage{hyperref}
\usepackage{makeidx}
\usepackage{tocloft}
\usepackage[utf8]{inputenc}
\usepackage{newunicodechar}
\usepackage{luacode}
\usepackage{titling}
\usepackage[inline]{enumitem}


\protected\def\pdfinfo {\pdfextension info }

\pdfinfo{
  /CreationDate (D:20170728090000-07'00')
  /ModDate (D:20170728090000-07'00')
  /Author (Lightning | Legitimate Business Syndicate)
  /Title (The cLEMENCy Computer Architecture)
  /Subject (computers)
}
\edef\pdftrailerid {\pdfvariable trailerid}
\pdftrailerid{[<6c65676974627320> <6c65676974627320>]}
\title{cLEMENCy Errata}
\date{July 2017}
\author{Lightning\\ Legitimate Business Syndicate}

\newunicodechar{←}{\leftarrow}

\addtolength{\cftsecnumwidth}{10pt}

\begin{document}
\frontmatter
\thispagestyle{plain}
\begin{center}
  \centering
      {
        \LARGE \thetitle
      }
      \\[2em]
      {
          \Large Legitimate Business Syndicate
      }
      \\[1em]

      \begin{itemize*}[label=]
      \item Deadwood
      \item Fuzyll
      \item Gyno
      \item HJ
      \item Hoju
      \item Jetboy
      \item Jymbolia
      \item Lightning
      \item Selir
      \item Sirgoon
      \item Thing2
      \item Vito
      \end{itemize*}
\end{center}

      To the extent possible under law, Legitimate Business Syndicate have waived all
      copyright and related or neighboring rights to ``\thetitle''. This work is
      published in the United States. For more information about this license, please
      visit https://creativecommons.org/publicdomain/zero/1.0/

      Compiled from revision
      \texttt{\input{rev.tex}} on \input{date.tex}.

\newpage
\mainmatter

\setcounter{chapter}{1}
\setcounter{section}{1}

\section{Stack}

\textit{The stack pointer change when an interrupt fires or is returned from has
  changed.}

Due to no specific stack based instructions, there is no expected direction the stack should grow, however during an interrupt the processor will subtract 99 bytes of data from the stack pointer when storing all registers to the stack and the interrupt return will read 99 bytes from the current stack pointer. If an implementation needs the interrupts to add to the stack instead of subtracting then changing the 'Interrupt stack direction flag' bit in the features area of the processor will accomplish this.

\setcounter{chapter}{2}

\setcounter{section}{1}

\section{Clock IO}

\textit{The zero time for the real-time clock has been changed.}

There are 6 bytes per timer with 4 timers maximum. A 0 for the timer delay disables that specific timer. Each timer has a 1 millisecond accuracy.





\begin{table}[H]
\centering
\caption{Clock and Timer Parameters}
\begin{tabu} spread 0pt{ X[-1c]  X[-2c]  X[l] }
Memory Start & Bytes & Details \\
\hline
\texttt{4000000} & \texttt{3} & Timer 1 Delay \\
\texttt{4000003} & \texttt{3} & Number of milliseconds left for Timer 1 \\
\texttt{4000006} & \texttt{3} & Timer 2 Delay \\
\texttt{4000009} & \texttt{3} & Number of milliseconds left for Timer 2 \\
\texttt{400000C} & \texttt{3} & Timer 3 Delay \\
\texttt{400000F} & \texttt{3} & Number of milliseconds left for Timer 3 \\
\texttt{4000012} & \texttt{3} & Timer 4 Delay \\
\texttt{4000015} & \texttt{3} & Number of milliseconds left for Timer 4 \\
\texttt{4000018} & \texttt{6} & Number of seconds since Aug. 02, 2013 09:00 PST \\
\texttt{400001B} & \texttt{3} & Number of processing ticks since processor start \\
\end{tabu}
\end{table}

\setcounter{section}{10}
\newpage

\section{Processor Identification}

\textit{The original version of this section mis-stated the stack pointer change
  when an interrupt fires or is returned from.}


The last 128 bytes of memory are used for processor identification and information of supported functionality. Writes to this area are ignored with the exception of the Interrupt Stack Direction Flag.





\begin{table}[H]
\centering
\caption{Processor Identification Attributes}
\begin{tabu} spread 0pt{ X[-1c]  X[-2c]  X[l] }
Memory Start & Bytes & Details \\
\hline
\texttt{7FFFF80} & \texttt{20} & Processor name \\
\texttt{7FFFFA0} & \texttt{3} & Processor version \\
\texttt{7FFFFA3} & \texttt{3} & Processor functionality flags \\
\texttt{7FFFFA6} & \texttt{4A} & For future use \\
\texttt{7FFFFF0} & \texttt{1} & Interrupt stack direction flag \\
\texttt{7FFFFF1} & \texttt{F} & For future use \\
\end{tabu}
\end{table}


It is recommended to implementors that the processor version contains a Major, Minor, and Revision value, one value per byte entry.


The functionality flags has the following information:


\begin{table}[H]
\centering
\caption{Processor Functionality Flags}
{
\tabulinesep = 2pt
\setlength{\tabcolsep}{2.5pt}
\begin{tabu} to \linewidth { X[r] X[r] X[r] X[r] X[r] X[r] X[r] X[r] X[r] X[r] X[r] X[r] X[r] X[r] X[r] X[r] X[r] X[r] X[r] X[r] X[r] X[r] X[r] X[r] X[r] X[r] X[r] }
\multicolumn{1}{r}{0} & \multicolumn{1}{r}{0} & \multicolumn{1}{r}{0} & \multicolumn{1}{r}{0} & \multicolumn{1}{r}{0} & \multicolumn{1}{r}{0} & \multicolumn{1}{r}{0} & \multicolumn{1}{r}{0} & \multicolumn{1}{r}{0} & \multicolumn{1}{r}{0} & \multicolumn{1}{r}{0} & \multicolumn{1}{r}{0} & \multicolumn{1}{r}{0} & \multicolumn{1}{r}{0} & \multicolumn{1}{r}{0} & \multicolumn{1}{r}{0} & \multicolumn{1}{r}{0} & \multicolumn{1}{r}{0} & \multicolumn{1}{r}{0} & \multicolumn{1}{r}{0} & \multicolumn{1}{r}{0} & \multicolumn{1}{r}{0} & \multicolumn{1}{r}{0} & \multicolumn{1}{r|}{0} & \multicolumn{1}{r|}{\cellcolor{lightgray!25}X} & \multicolumn{1}{r|}{X} & \multicolumn{1}{r|}{\cellcolor{lightgray!25} X} \\
\multicolumn{23}{r}{} & \multicolumn{1}{r|}{} & \multicolumn{1}{r|}{\cellcolor{lightgray!25}} & \multicolumn{1}{r|}{} & \multicolumn{1}{r|}{\cellcolor{lightgray!25}} \\
\cline{1-24}
\multicolumn{25}{r|}{\cellcolor{lightgray!25} Interrupts are able to flip stack storage direction} & \multicolumn{1}{r|}{} & \multicolumn{1}{r|}{\cellcolor{lightgray!25}} \\
\cline{1-25}
\multicolumn{26}{r|}{FPU built into the processor} & \multicolumn{1}{r|}{\cellcolor{lightgray!25}} \\
\cline{1-26}
\multicolumn{27}{r|}{\cellcolor{lightgray!25} Processor supports 54-bit math} \\
\cline{1-27}
\end{tabu}
\\
}
\end{table}



The low bit of the interrupt stack direction flag dictates the direction the interrupt writes to the stack. A value of 0, the default, results in the interrupt subtracting 99 bytes from the stack pointer then storing all registers in the 99 byte buffer starting with register 0. A value of 1 results in the interrupt storing and incrementing the stack pointer starting with register 0. The interrupt return behaves in the opposite manner to restore the registers.

\setcounter{chapter}{3}
\setcounter{section}{-1}
\newpage
\section{Interrupts and Exceptions}

\textit{The original version of the introduction to this chapter mis-stated what
  is pushed to the stack when an interrupt fires or is returned from.}

When any interrupt is fired, all 32 general purpose registers and low 4 bits of the flags register are stored to the current stack before the processor begins executing the specified interrupt routine. Upon returning from an interrupt, all registers and low bits of the flag register are restored from the stack. The Disable Interrupts, DI, and Enable Interrupts, EI, instructions are used to temporarily disable an interrupt. Any interrupt with a value of 0 will not be called and ignored.


\setcounter{section}{2}


\section{Divide by 0 Exception}

\textit{This interrupt now fires on all floating point exceptions, in addition
  to division by zero.}

Division with a divisor of 0 will trigger this interrupt. Additionally, all other floating point exceptions will also trigger this interrupt.

\setcounter{chapter}{4}
\setcounter{section}{46}

\section{IR: Interrupt Return}

\textit{The flags are now pushed to the stack, and the increment amount on the
  stack pointer has been changed to reflect this.}

\index{IR}
{
\setlength{\tabcolsep}{3pt}
\begin{tabu} spread \linewidth {l r}
\multicolumn{1}{l}{0} & \multicolumn{1}{r}{17} \\
\multicolumn{2}{c}{\texttt{101000000001000000}}
\end{tabu}
}
\nopagebreak
\begin{description}
\item [Format:] \texttt{IR}
\item [Purpose:] Return from an interrupt
\item [Description:] Return from an interrupt routine.

\item [Operation:] \begin{verbatim}

if Stack_Direction_Flag then
  SP -= 0x63
R0:R31 ← [SP:SP+0x60]
FLAGS ← [SP+0x60:SP+0x63]
if not Stack_Direction_Flag then
  SP += 0x63\end{verbatim}
\item [Flags affected:] \textit{Z C O S}
\end{description}

\end{document}
